\documentclass[11pt]{beamer}
\geometry{papersize={16cm,9cm}}
\usetheme{default}
\usecolortheme{beaver}
\setbeamertemplate{navigation symbols}{}

\title{Example document to recreate with beamer in \LaTeX}
\setbeamerfont{title}{size=\fontsize{14}{14}}
\subtitle{}
\author{Your Name}
\date{}


\begin{document}

%--- the titlepage frame -------------------------%
\begin{frame}[plain]
  \vspace*{1.5cm}\titlepage  
  \vspace*{1cm}
  \centering Fall 2019 \\ Markup Languages and Reproducible Programming in Statistics\par
\end{frame}

%--- slide 1 -------------------------%
\begin{frame}{Outline}

Working with equations \\
  \hspace*{0.5cm}
  Aligning the same equations \\
  \hspace*{0.5cm}
  Omit equation numbering \\
  \hspace*{0.5cm}
  Ugly alignment \\
\bigskip
\bigskip
Discussion

\end{frame}

%--- slide 2 -------------------------%
\begin{frame}{Working with equations}

We define a set of equations as
\begin{gather}
a = b + c^2, \\
a - c^2 = b, \\
\text{left side} = \text{right side}, \\
\text{left side} + \text{something} \geq \text{right side}, 
\end{gather}
for all something $>$ 0.

\end{frame}

%--- slide 3 -------------------------%
\begin{frame}{Aligning the same equations}

Aligning the equations by the equal sign gives a much better view into the placements of the separate equation components.
\begin{align}
a &= b + c^2, \\
a - c^2 &= b, \\
\text{left side} &= \text{right side}, \\
\text{left side} + \text{something} &\geq \text{right side}, 
\end{align}

\end{frame}

%--- slide 4 -------------------------%
\begin{frame}{Omit equation numbering}

Alternatively, the equation numbering can be omitted.
\begin{align*}
a &= b + c^2, \\
a - c^2 &= b, \\
\text{left side} &= \text{right side}, \\
\text{left side} + \text{something} &\geq \text{right side}, 
\end{align*}

\end{frame}

%--- slide 5 -------------------------%
\begin{frame}{Ugly alignment}

Some components do not look well, when aligned. Especially equations with different heights and spacing. For example,
\begin{align}
E &= m c^2, \\
m &= \frac{E}{c^2}, \\
c &= \sqrt{\frac{E}{m}}. 
\end{align}
Take that into account.

\end{frame}

%--- slide 6 -------------------------%
\begin{frame}{Discussion}

This is where you'd normally give your audience a recap of your talk, where you could discuss e.g. the following
\begin{itemize}
  \item Your main findings
  \item The consequences of your main findings
  \item Things to do
  \item Any other business not currently investigated, but related to your talk
\end{itemize}

\end{frame}

\end{document}
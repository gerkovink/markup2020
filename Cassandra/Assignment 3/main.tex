\documentclass[aspectratio=169]{beamer}
%
% Choose how your presentation looks.
%
% For more themes, color themes and font themes, see:
% http://deic.uab.es/~iblanes/beamer_gallery/index_by_theme.html
%
\mode<presentation>
{
  \usetheme{default}      % or try Darmstadt, Madrid, Warsaw, ...
  \usecolortheme{beaver} % or try albatross, beaver, crane, ...
  \usefonttheme{default}  % or try serif, structurebold, ...
  \setbeamertemplate{navigation symbols}{}
  \setbeamertemplate{caption}[numbered]
} 

\usepackage[english]{babel}
\usepackage[utf8]{inputenc}
\usepackage[T1]{fontenc}

\title[Your Short Title]{Example document to recreate with \texttt{beamer} in \LaTeX{}}
\author{Cassandra Bunschoten}
\vskip 10 cm
\date{FALL 2019 \\
Markup Languages and Reproducible Programming in Statistics}



\begin{document}

\begin{frame}
  \titlepage
\end{frame}

% Outline
\begin{frame}{Outline}
\tableofcontents
\end{frame}

%first section, equations
\section{Working with equations}

%slide equations
\begin{frame}{Working with equations}

\vskip 1cm
\
We define a set of equations as
\begin{equation}
a = b + c^2,
\end{equation}
\begin{equation}
a - c^2 = b,
\end{equation}
\begin{equation}
    \text{left side $-$ right side,}
\end{equation}
\begin{equation}
    \text{left side + something $\geq$ right side},
\end{equation}
\
for all something > 0.

\vskip 2cm

\end{frame}

%slide aligning equations
\subsection{Aligning the same equations}

%slide equations
\begin{frame}{Aligning the same equations}

\vskip 1cm
Aligning the equations by the equal sign gives a much better view into the placements of the separate equation components.

\begin{eqnarray}
a &=& b + c^2 \\
a - c^2 &=& b \\
\text{left side} &-& \text{right side} \\
\text{left side + something} &\geq& \text{right side}
\end{eqnarray}

\vskip 2cm

\end{frame}


\subsection{Omit equation numbering}

\begin{frame}{Omit equation numbering}

Alternatively, the equation numbering can be omitted.
\begin{eqnarray*}
a &=& b + c^2 \\
a - c^2 &=& b \\
\text{left side} &-& \text{right side} \\
\text{left side + something} &\geq& \text{right side}
\end{eqnarray*}

\end{frame}

\subsection{Ugly alignment}

\begin{frame}{Ugly alignment}

Some components do not look well, when aligned. Especially equations with different heights and spacing. For example,

\begin{eqnarray}
E &=& mc^2, \\
m &=& \frac{E}{c^2}, \\
c &=& \sqrt{\dfrac{E}{m}}.
\end{eqnarray}

Take that into account.

\end{frame}


\section{Discussion}
\begin{frame}{Discussion}

This is where you'd normally give your audience a recap of your talk, where you could discuss e.g. the following

\begin{itemize}
    \item Your main findings
    \item The consequences of your main findings
    \item Things to do
    \item Any other business not currently investigated, but related to your talk
\end{itemize}
    
\end{frame}


\end{document}

\documentclass[aspectratio = 169]{beamer}
\usetheme{default}
\usecolortheme{beaver}
\beamertemplatenavigationsymbolsempty

\usepackage{amsmath}

% Slide title page
\title{Example document to recreate with \texttt{beamer} in \LaTeX}
\author{Zo\"e Dunias}
\date{\vspace{.5 in}\\FALL 2020\\ Markup Languages and Reproducible Programming in Statistics \vskip6mm}

\begin{document}
\titlepage

% Slide outline
\begin{frame}{Outline}

\tableofcontents

\end{frame}

% Slide working with equations
\section{Working with equations}
\begin{frame}{Working with equations}

We define a set of equations as
\begin{equation}
a = b + c^2,
\end{equation}
\begin{equation}
a - c^2 = b,
\end{equation}
\begin{equation}
\text{left side} = \text{right side},
\end{equation}
\begin{equation}
\text{left side} + \text{something} \geq \text{right side},
\end{equation}
for all something $>$ 0.

\end{frame}

% Slide aligning the same equations
\subsection{Aligning the same equations}
\begin{frame}{Aligning the same equations}

Aligning the equations by the equal sign gives a much better view into the placements of the separate equation components.
\begin{align}
a &= b + c^2,\\
a - c^2 &= b,\\
\text{left side} &= \text{right side},\\
\text{left side} + \text{something} &\geq \text{right side},
\end{align}

\end{frame}

% Slide omit equation numbering
\subsection{Omit equation numbering}
\begin{frame}{Omit equation numbering}

Alternatively, the equation numbering can be omitted.
\begin{align*}
a &= b + c^2\\
a - c^2 &= b\\
\text{left side} &= \text{right side}\\
\text{left side} + \text{something} &\geq \text{right side}
\end{align*}

\end{frame}

% Slide ugly alignment
\subsection{Ugly alignment}
\begin{frame}{Ugly alignment}

Some components do not look well, when aligned. Especially equations with different heights and spacing. For example,
\begin{align}
E &= mc^2,\\
m &= \frac{E}{c^2},\\
c &= \sqrt{\frac{E}{m}}.
\end{align}
Take that into account.

\end{frame}

% Slide discussion
\section{Discussion}
\begin{frame}{Discussion}

This is where you’d normally give your audience a recap of your talk, where you could
discuss e.g. the following
\begin{itemize}
\item Your main findings
\item The consequences of your main findings
\item Things to do
\item Any other business not currently investigated, but related to your talk
\end{itemize}

\end{frame}

\end{document}
% Specify document class, themes and packages
\documentclass{beamer}
\usetheme{default}
\usecolortheme{beaver}
\usepackage{amsmath}

% Specify title, author and data+ subtext of date 
\title{Exercise 3: Beamer}
\author{Max Van der Velde }
\date{\vspace{1 in}\\November 23, 2020\\ Markup Languages and Reproducible Programming in Statistics }
%%% ACTUAL START OF THE DOCUMENT
\begin{document}
\maketitle
\begin{frame}{Outline}
\tableofcontents
\end{frame}

% Start primary section
\section{working with equations}
\begin{frame}{Working with equations}
We define a set of equations as
\begin{equation}
    a=b+c^2,
\end{equation}
\begin{equation}
    a-c^2=b,
\end{equation}
\begin{equation}
    \text{left side = right side,}
\end{equation}
\begin{equation}
    \text{left side} +\text{something} \geq \text{right side} 
\end{equation}
for all something $> 0$.
\end{frame}

% Start subsection of section
\subsection{Aligning the same equations}
\begin{frame}{Aligning the same equations}
Aligning the equations by the equal sign gives a much better view into the placements of the separate equation components.
\begin{equation}\hspace{1 in}
    a=b+c^2,
\end{equation}
\begin{equation}\hspace{0.35 in}
    a-c^2=b,
\end{equation}
\begin{equation}\hspace{0.8 in}
    \text{left side = right side,}
\end{equation}
\begin{equation}\hspace{-0.1 in}
    \text{left side} +\text{something} \geq \text{right side} 
\end{equation}
\end{frame}

%%% SUBSECTION 2!
\subsection{Omit equation numbering}
\begin{frame}{Omit equation numbering}
Alternatively, the equation numbering can be omitted.
\begin{equation*}\hspace{1 in}
    a=b+c^2
\end{equation*}
\begin{equation*}\hspace{0.35 in}
    a-c^2=b
\end{equation*}
\begin{equation*}\hspace{0.8 in}
    \text{left side = right side}
\end{equation*}
\begin{equation*}
    \text{left side} +\text{something} \geq \text{right side} 
\end{equation*}
\end{frame}

%% SUBSECTION 3
\subsection{Ugly...alignment}
\begin{frame}{Ugly alignment}
Some components do not look well, when aligned. Especially equations with different heights and spacing. For example,
\begin{equation}
    E=mc^2,
\end{equation}
\begin{equation}
m=\frac{E}{c^2},
\end{equation}
\begin{equation}
    c= \sqrt{\frac{E}{m}}.
\end{equation}
Take that into account
\end{frame}

%% final section
\section{Discussion}
\begin{frame}{Discussion}
This is where you’d normally give your audience a recap of your talk, where you could discuss e.g. the following
\begin{itemize}
    \item Your main findings
    \item The consequences of your main findings
    \item Things to do
    \item Any other business not currently investigated, but related to your talk
\end{itemize}
\end{frame}
\end{document}

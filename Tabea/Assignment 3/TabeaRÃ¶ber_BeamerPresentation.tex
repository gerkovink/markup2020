\documentclass[aspectratio=169]{beamer}

\usetheme[]{default}
\usecolortheme{beaver}

\beamertemplatenavigationsymbolsempty %suppress navigation bar
\setbeamersize{description width=0.57cm}
\definecolor{light-gray}{gray}{0.1}

\usepackage{amsmath}

%TITLEPAGE
\title[]{Example document to recreate with \texttt{beamer} in \LaTeX}
\author[T. Röber]{Tabea Röber}

\date[]{\vspace{0.5 in}\\ FALL 2020 \\ Markup Languages and Reproducible Programming in Statistics  \vskip6mm}



\begin{document}
%--- the titlepage frame -------------------------%
\begin{frame}[plain]
\vspace{0.5 in}
  \titlepage
\end{frame}

%--- slide 1: outline ----------------%
% note: this could also be done using the \tableofcontents and \sections and \subsections commands
\begin{frame}{Outline}
Working with equations
  \begin{description}
    \item Aligning the same equations
    \item Omit equation numbering
    \item Ugly alignment
  \end{description}
 \vspace{0.4 in}
Discussion
\end{frame}



%--- slide 2: working with equations ----------------%
\begin{frame}
  \frametitle{Working with equations}
We define a set of equations as

\begin{equation}
a = b + c^2, 
\end{equation}
\begin{equation}
a - c^2 = b, 
\end{equation}
\begin{equation}
\text{left side} = \text{right side,} 
\end{equation}
\begin{equation}
\text{left side} + \text{something} \geq \text{right side,}
\end{equation}

for all something $ > 0 $.
 \end{frame}


%--- slide 3: aligning the same equations----------------%
\begin{frame}
  \frametitle{Aligning the same equations}
Aligning the equations by the equal sign gives a much better view into the placements of the separate equation components.

\begin{align}
a &= b + c^2, \\
a - c^2 &= b, \\
\text{left side} &= \text{right side,} \\
\text{left side} + \text{something} &\geq \text{right side,}
\end{align}

 \end{frame}



%--- slide 4: omit equation numbering----------------%
\begin{frame}
  \frametitle{Omit equation numbering}
Alternatively, the equation numbering can be omitted.

\begin{align*}
a &= b + c^2, \\
a - c^2 &= b, \\
\text{left side} &= \text{right side,} \\
\text{left side} + \text{something} &\geq \text{right side,}
\end{align*}

 \end{frame}

%--- slide 5: ugly alignment----------------%
\begin{frame}
  \frametitle{Ugly alignment}
Some components do not look well, when aligned. Especially equations with different heights and spacing. For example,

\begin{align}
E &= mc^2, \\
m &= \frac{E}{c^2}, \\
c &= \sqrt{\frac{E}{m}} 
\end{align}
Take that into account.

 \end{frame}

%--- slide 6: discussion----------------%
\begin{frame}
  \frametitle{Discussion}
This is where you’d normally give your audience a recap of your talk, where you could discuss e.g. the following

\begin{itemize}
\item Your main findings
\item The consequences of your main findings
\item Things to do 
\item Any other business not currently investigated, but related to your talk
\end{itemize}

 \end{frame}



\end{document}
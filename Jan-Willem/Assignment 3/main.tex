\documentclass[8pt]{beamer}
\usetheme{Pittsburgh}
\usecolortheme{beaver}
\setbeamertemplate{frametitle}[default]
\geometry{paperheight = 200 pt} 

\title{Example document to recreate with beamer in \LaTeX}
\author{J.W.G. Simons}
\date{}

\begin{document}

\begin{frame}
\titlepage
\centering
\text{FALL 2019}
\linebreak
\text{Markup Languages and Reproducible Programming in Statistics}
\end{frame}

\begin{frame}{Outline}

\hyperlink{wwe}{Working with equations}
\begin{itemize}
  \item[] \hyperlink{atse}{Aligning the same equations}
  \item[] \hyperlink{oen}{Omit equation numbering}
  \item[] \hyperlink{ua}{Ugly alignment}
\end{itemize}
\bigskip
\bigskip
\hyperlink{disc}{Discussion}


\end{frame}

\begin{frame}[label = wwe]{Working with equations}
We define a set of equations as 
\begin{equation}
 a = b + c^2, 
\end{equation}
\begin{equation}
 a - c^2 = b, 
\end{equation}
\begin{equation}
\text{left side = right side,} 
\end{equation}
\begin{equation}
\text{left side + something $\geq$ right side,}
\end{equation}
for all something $>$ 0.
\end{frame}

\begin{frame}[label = atse]{Aligning the same equations}

Aligning the equations by the equal sign gives a much better view into the placements of the separate equation components.

\begin{align} 
a &= b + c^2, \\ 
a - c^2 &= b, \\
\text{left side} &= \text{right side} \\ 
\text{left side + something} &\geq \text{right side,}
\end{align}

\end{frame}

\begin{frame}[label = oen]{Omit equation numbering}

Alternatively, the equation numbering can be omitted.

\begin{align*} 
a &= b + c^2, \\ 
a - c^2 &= b, \\
\text{left side} &= \text{right side} \\ 
\text{left side + something} &\geq \text{right side,}
\end{align*}

\end{frame}

\begin{frame}[label = ua]{Ugly alignment}

Some components do not look well, when aligned. Especially equations with different heights and spacing. For example,

\begin{align} 
E &= mc^2, \\ 
m &= \frac{E}{c^2}, \\
c &= \sqrt{\frac{E}{M}}
\end{align}

Take that into account.

\end{frame}

\begin{frame}[label = disc]{Discussion}

This is where you’d normally give your audience a recap of your talk, where you could discuss e.g. the following

\setbeamertemplate{itemize items}[triangle]
\begin{itemize}
 \item Your main findings
 \item The consequences of your main findings
 \item Things to do
 \item Any other business not currently investigated, but related to your talk
\end{itemize}

\end{frame}

\end{document}